\documentclass{idc_msc}

\title{Cryptography \\\large Lecture 10}
\date{2019-05-28 \\ Last edited \currenttime\ \today}
\author{Lecture by Dr. Alon Rosen\\Typeset by Steven Karas}

\AtBeginDocument{\maketitle}
\AtEndDocument{\vfill\bibliographystyle{plain}\bibliography{../biblio}{}}
\let\E\relax
\DeclareMathOperator*{\E}{\mathrm{E}}
\DeclareMathOperator*{\xor}{\oplus}
\let\NPclass\relax
\newcommand{\NPclass}{\texorpdfstring{\ensuremath{\mathcal{NP}}}{NP}}
\let\Pclass\relax
\newcommand{\Pclass}{\texorpdfstring{\ensuremath{\mathcal{P}}}{P}}
\let\Cipherspace\relax
\newcommand{\Cipherspace}{\texorpdfstring{\ensuremath{\mathcal{C}}}{C}}
\let\Messagespace\relax
\newcommand{\Messagespace}{\texorpdfstring{\ensuremath{\mathcal{P}}}{P}}
\let\Keyspace\relax
\newcommand{\Keyspace}{\texorpdfstring{\ensuremath{\mathcal{K}}}{K}}
\let\getsrandom\relax
\newcommand{\getsrandom}{\ensuremath{\overset{R}{\gets}}}
\let\ints\relax
\newcommand{\ints}{\ensuremath{\mathbb{Z}}}
\let\reals\relax
\newcommand{\reals}{\ensuremath{\mathbb{Z}}}
\let\negligible\relax
\DeclareMathOperator*{\negligible}{\mathrm{neg}}
\let\polynomial\relax
\DeclareMathOperator*{\polynomial}{\mathrm{poly}}
\let\divides\relax
\newcommand{\divides}[1]{{}^{#1}\vert}
\let\ndivides\relax
\newcommand{\ndivides}[1]{{}^{#1}\not\vert}

\begin{document}

\paragraph{Disclaimer}

These notes are based on the lectures for the course Cryptography, taught by Dr. Alon Rosen at IDC Herzliyah in the spring semester of 2018/2019.
Sections may be based on the lecture slides prepared by Dr. Alon Rosen.

\nocite{Katz:2014:IMC:2700550}

\section{Recap}

I was not present in the last three lectures due to scheduling conflicts.

\section{Agenda}

\begin{itemize}
  \item MAC
\end{itemize}

\section{Message Authentication (MAC)}

Note that encryption in and of itself does not provide authentication.
The rough idea is to attach a "tag" to the message to prove that the message is authentic.
MAC provides the symmetric encryption equivalent of asymmetric signatures.
Note that asymmetric signatures can be verified by anyone, whereas MACs can only be authenticated by those who know the key.

For example if Alice wants to send a message to Bob where they both have a shared key \(k\), Alice can send \((m, t)_k\) to Bob who can verify that the message tag corresponds to the key \(k\).
The security of this authentication is limited by knowledge of the key.

\subsection{Definition}

A MAC consists of three algorithms \((G, M, V)\):

\begin{enumerate}
  \item A key generation algorithm \(k \getsrandom G(1^n)\)
  \item A tagging algorithm \(t \getsrandom M_k(m)\)
  \item A verification algorithm \(V_k(m, t) \in \{\mathrm{ACC}, \mathrm{REJ}\}\)

  % note that a verifier that always accepts is syntactically correct
  \[
  \forall m \; V_k(m, M_k(m)) = \mathrm{ACC}
  \]
\end{enumerate}

% These definitions were presented by McCallister, Iverson, and Rivest in 1984
% Between Diffie-Hellman and then, conventional wisdom was that signatures cannot possibly be secure.

\paragraph{Adversarial Model}

Given an adversary \(F\), their goal is to produce a valid tag for message they have never seen a valid tag for.
Assume that \(F\) can request from an oracle valid tags \(t_i = M_k(m_i)\) for any number of messages \(m_i\).
We say that \(F\) forges \(V_k(m, M_k(m)) = \mathrm{ACC}\) for some \(m \ne m_i \forall i\).

\subsection{Existential unforgeability under adaptive chosen message attack}

A MAC \((G,M,V)\) is secure if for any PPT \(F\):

\[
  \exists \negligible \varepsilon : \Pr\left[ F^{M_k(\cdot)}(1^n) \; \textrm{forges} \right] \le \varepsilon
\]

% if we use a serial number for subsequent tags of a message m, then the definitions are largely the same, but M_k needs to drag the state along with it.

\subsection{Fixed length MAC}

\(M_k(m) = f_k(m)\) where \(f_k\) is a PRF (\(F_n=\{f_k:\{0,1\}^n \to \{0,1\}^n\}\)), where \(V_k(m, t) \in \{\mathrm{ACC}, \mathrm{REJ}\}\).

If \(F = U_n F_n\) is a PRF family then the MAC described above is secure.
Let \(F\) be a PPT forger.

\[
  \Pr[F^{R}\; \textrm{forges}] \le 2^{-n}
\]
\[
  \Pr[F^{f_k} \; \textrm{forges}] \le 2^{-n} + \negligible
\]

If such a forger were to exist, then it would be able to decide between truly random and pseudorandom.

\subsection{Arbitrary length MAC}

Given a \((G,M,V)\) that is a fixed length MAC for a blocks of size \(n\).

\paragraph{Attempt 1:}

\[
  M'_k(m_1,\ldots,m_d) = M_k(m_1 \xor \cdots \xor m_d)
\]

However, this is equivalent to:

\[
  M'_k(m_2,m_1, \ldots, m_d)
\]

which is a message that was never sent.

\paragraph{Attempt 2:}

\[
  M'_k(m_1,\ldots,m_d) = (M_k(m_1), \ldots, M_k(m_d))
\]

However, this is also insecure because we can construct:

\[
  M'_k(m_2, m_1,\ldots,m_d) = (M_k(m_2), M_k(m_1), \ldots, M_k(m_d))
\]

\paragraph{Attempt 3:}

\[
  M'_k(m_1, \ldots, m_d) = M_k(m_1, 1), \ldots, M_k(m_d, d)
\]

Where each block \(m_1\) has size \(n/2\).

However, if we take two different messages from the oracle, we can mix and match between the messages to construct a valid tag.

\paragraph{Attempt 4:}

Consider \(M'_k(m)\) where \(m = m_1, \ldots, m_d\) where \(m_i \in \{0,1\}^{n/3}\).
Let \(r \getsrandom \{0, 1\}^{n/3}\).
Let \(t_i \gets M_k(m_i, i, d, r)\)
Output \(r,t_1, \ldots, t_d\)

Let \(V'_k(r,t_1, \ldots, t_d) = \mathrm{ACC}\) iff \(\forall i \in \{1, \ldots, d\}\) it holds that \(V_k((m_i, i, d, r), t_i) = \mathrm{ACC}\).

If \((G, M, V)\) is secure then \((G', M', V')\) is secure.
Suppose that \(F'\) is a PPT that forges \((G', M', V')\) with probability \(\varepsilon\).
\(F'\) gets MAC for polynomially many messages.
If \(F'\) is successful then \(m=m_1, \ldots, m_d\), \(t= (r, t_1, \ldots, t_d)\).

\[
  \forall i \;\; V_k((m_i, i, d, r), t_i) = \mathrm{ACC}
\]

We have two cases.
In the first case \(t = (r, t_1, \ldots, t_d)\) contains a "new" \(r\).
This means that \((m_i, i, d, r)\) was never previously authenticated with \(M_k\).
In the second case \(t\) does not contain a "new" \(r\).
Either \(r\) has appeared in only one previous MAC, in which case it would be equivalent to breaking \(M_k\).
The other possibility is that \(r\) has appeared in more than one previous MAC, which happens with probability \(q^2/ 2^{n/3}\).

\paragraph{CBC MAC}

size of tag is \(n\) (not \(dn + n/3\)).
Let \(F = \{f_k : \{0, 1\}^n \to \{0, 1\}^n\}\) be a family of PRFs.
Let \(M_k(m_1, \ldots, m_d) = y_d\) where \(y_i = f_k(m_i \xor y_{i-1})\) and \(y_0 = 0^n\).

CBC MAC is secure for \(m \in \{0, 1\}^{dn}\).

\section{Digital Signatures}

\[
  m^{sk \;A^{vk}} \overset{m,S_{sk}(m)}{\longrightarrow}^{B} V_{vk}(m,S_{sk}(m)) = \mathrm{ACC}
\]

which means that Alice signs a message using a signing key \(sk\), sends that signature and the message to Bob, who knows the corresponding verification key \(vk\) who can then verify that the message was signed by \(sk\).

This is publicly verifiable (\(vk\)) is not secret.
It is transferable.
It is also non-repudiable.

\subsection{Syntax}

A digital signature consists of three algorithms \((G,S,V)\).

\begin{enumerate}
  \item A key generator \((vk, sk) \getsrandom G(1^n)\)
  \item A signature algorithm \(\sigma \getsrandom S_{sk}(m)\)
  \item a verification algorithm \(V_{vk}(m,\sigma) \in \{\mathrm{ACC}, \mathrm{REJ}\}\)
\end{enumerate}

We require that \(V_{vk}(m, S_{sk}(m)) = \mathrm{ACC}\) for any message \(m\).

Note that the sender is the one who needs \(sk\), whereas in the PKE it is the receiver.
However, these are \textbf{not} the dual of PKE.
Decryption does not imply signing; notably, \(x^2 \bmod n\).
Randomization is not necessary.

Security is defined exactly as in MACs, but where the adversary already has the verification key.

\subsection{Applications}

\begin{enumerate}
  \item Public key infrastructure requires trust roots who certify that a given key belongs to an entity.
  For example, Verisign will provide Alice with a certificate that her keys are actually hers.
  When someone begins communicating with her, they can verify that the party on the other end actually has Alice's keys, assuming they trust Verisign.
  \item Blockchains are built on the concept that a transfer is signed by the originating keypair that something has been transferred.
\end{enumerate}

\subsection{Example constructions}

\paragraph{Attempt 1}

Let \(f\) be a OWF such that:

\begin{enumerate}
  \item \(y=f(x)\) where \(sk = x\), \(vk = y\), and \(x \getsrandom \{0,1\}^n\)
  \item \(S_{sk}(m) = x\)
  \item \(V_{vk}(m,\sigma) = 1 \Leftrightarrow f(\sigma) = y\)
\end{enumerate}

\paragraph{Attempt 2}

Let \(F = \{f_i : D_i \to D_i\}\) be a family of TDPs\footnote{Trap-door-permutations}.

\begin{enumerate}
  \item \(vk = i\), \(sk = t\)
  \item \(S_{sk}(m) = f^{-1}(m)\)
  \item \(V_{vk}(m, \sigma)\) checks that \(f_i(\sigma) = m\)
\end{enumerate}

However, we can attack this without seeing any messages by showing that for \(m=1\), \(\sigma=1\) if we choose RSA as the trapdoor function.
Another attack is to choose some signature \(\sigma \in D\) and then compute the corresponding message \(m = f_i(\sigma)\).

\paragraph{Attempt 3}

Textbook RSA.

\begin{enumerate}
  \item \(vk = (N, d)\), \(sk = (N, e)\)
  \item \(S_{sk}(m) = m^e \mod N\)
  \item \(V_{pk}(m, \sigma) = 1 \Leftrightarrow \sigma^d = m \mod N\)
\end{enumerate}

However, note that \((m \cdot m')^e = m^e m'^e \mod N\). To forge, we just need to ask for \(\sigma_1, \sigma_2\) corresponding to \(m_1, m_2\) where \(m_2 = m_1^{-1} \cdot m \mod N\).
\(\sigma_1 \cdot \sigma_2\) is a valid signature on \(m_1 \cdot m_2 = m_1 \cdot m_1^{-1} \cdot m = m\).

\paragraph{Attempt 4}

As a variation on this, we can use what is called a full domain hash, which requires a hash function \(H\) that breaks the homomorphisms such as \((m \cdot m')^e = m^e m'^e \mod N\).

\begin{enumerate}
  \item \(vk = (N, d)\), \(sk = (N, e)\)
  \item \(S_{sk}(m) = H(m)^e \mod N\)
  \item \(V_{pk}(m, \sigma) = 1 \Leftrightarrow \sigma^d = m \mod N\)
\end{enumerate}

\section{Next lecture}

\end{document}
