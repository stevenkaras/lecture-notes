\documentclass{idc_msc}

\title{Cryptography \\\large Lecture 5}
\date{2019-04-02 \\ Last edited \currenttime\ \today}
\author{Lecture by Dr. Alon Rosen\\Typeset by Steven Karas}

\AtBeginDocument{\maketitle}
\AtEndDocument{\vfill\bibliographystyle{plain}\bibliography{../biblio}{}}
\let\E\relax
\DeclareMathOperator*{\E}{\mathrm{E}}
\DeclareMathOperator*{\xor}{\oplus}
\let\NPclass\relax
\newcommand{\NPclass}{\texorpdfstring{\ensuremath{\mathcal{NP}}}{NP}}
\let\Pclass\relax
\newcommand{\Pclass}{\texorpdfstring{\ensuremath{\mathcal{P}}}{P}}
\let\Cipherspace\relax
\newcommand{\Cipherspace}{\texorpdfstring{\ensuremath{\mathcal{C}}}{C}}
\let\Messagespace\relax
\newcommand{\Messagespace}{\texorpdfstring{\ensuremath{\mathcal{P}}}{P}}
\let\Keyspace\relax
\newcommand{\Keyspace}{\texorpdfstring{\ensuremath{\mathcal{K}}}{K}}
\let\getsrandom\relax
\newcommand{\getsrandom}{\ensuremath{\overset{R}{\gets}}}
\let\negligible\relax
\DeclareMathOperator*{\negligible}{\mathrm{neg}}
\let\polynomial\relax
\DeclareMathOperator*{\polynomial}{\mathrm{poly}}
\let\divides\relax
\newcommand{\divides}[1]{{}^{#1}\vert}

\begin{document}

\paragraph{Disclaimer}

These notes are based on the lectures for the course Cryptography, taught by Dr. Alon Rosen at IDC Herzliyah in the spring semester of 2018/2019.
Sections may be based on the lecture slides prepared by Dr. Alon Rosen.

\nocite{Katz:2014:IMC:2700550}

\section{Recap}

Last week we covered pseudorandom generators that provide computational security:

An encryption scheme \((G,E,D)\) satisfies computational security if for any probabilistic polytime adversary \(A\) there exists some negligible function \(\varepsilon\) such that:

\[
  |\Pr[A(G(U_n)) = 1] - \Pr[A(U_{\ell(n)}) = 1| < \varepsilon(n)
\]

Where \(\ell(n)\) is the expansion of the original \(n\)-bit key of the PRG to \(\ell(n)\)-bit output.

\paragraph{PRGs as one-way functions}

PRGs can be considered to be one-way functions.
Define the following language:

\[
  L_G = \{y \in \{0, 1\}^{\ell(n)} \mid \exists x \in \{0, 1\}^n,\; G(x) = y\}
\]

Note that \(L_G\) is in \NPclass, because if we are given \(x\), it serves as a witness.

\section{Agenda}

\begin{itemize}
  \item One way functions
  \item Number theory
\end{itemize}

\section{One way functions}

A one-way function \(f\) is a function where:

\[
\begin{aligned}
  x &\overset{\text{"easy"}}{\longmapsto} f(x) \\
  f(x) &\overset{\text{"hard"}}{\longmapsto} f^{-1}(f(x))
\end{aligned}
\]

As such, when we are given \((x, f(x))\), we can easily verify the computation, yet when only given \(f(x)\), it is not easy.

\subsection{Formal Definition}

\(f: \{0, 1\}^* \to \{0, 1\}^*\) is a one-way function if:

\begin{enumerate}
  \item \(f\) can be evaluated in polytime\footnote{This is the asymptotic version. The concrete version is that a function is \(t, \varepsilon\) one-way function if it runs in concrete time \(t\)}.
  \item For any PPT \(A\) there exists a negligible \(\varepsilon\) such that:

  \[
    \Pr[A(f(U_n), 1^n) \in f^{-1}(f(U_n))] \le \varepsilon(n)
  \]
\end{enumerate}


\subsection{Inverting OWFs}

The following are sufficient definitions of 

\begin{enumerate}
  \item Find some preimage (not necessarily the same input as the original)
  \item with probability \(1 / \polynomial(n)\)
  \item For infinitely many \(n\)s
\end{enumerate}

Observations:

\begin{enumerate}
  \item \(|f(x) \le \polynomial(|x|)\)
  \item \(A(f(x)) = x\) is too demanding for A. e.g for \(f(x) = |x|\), \[\Pr[A(f(x)) = x] \le 2^{-n}\].
  \item We need to give \(A\) a second input: \(1^n\). For example, \(f(x) = |x|\) would not succeed because it wouldn't be able to even write \(x\), because \(\polynomial(|x|) \ll |x|\).
\end{enumerate}

\subsection{Candidate OWFs}
\marginpar{No one really has a good idea of a sufficient characteristic of a OWF, but low-degree functions are pretty well known to be bad.}

\subsubsection{Multiplication}

\[
  f(x, y) = x \cdot y \qquad |x| = |y|
\]

In this case, note that the inputs must be padded to conform to input length restrictions.

\paragraph{Factoring assumption}

For any PPT \(A\), there exists a negligible \(\varepsilon\) such that:

\[
  \Pr[A(N) \in \{P, Q\}] \le \varepsilon(n)
\]

Where \(P, Q\) are random \(n\)-bit primes and \(N=PQ\).

If the factoring assumption holds, then multiplication is a OWF.

\subsubsection{Subset sum}

\[
  f(\overset{\{0,1\}^n}{x_1}, \ldots, x_n, S) = (x_1, \ldots, x_n, \sum_{i \in S} x_i \mod 2^n)
\]

Note that because the expected unpadded length of any \(x_i\) is \(n-1\), we expect the sum to wrap around on average \(n/2\) times.

\subsubsection{Others}

\[
  f(k) = \mathrm{DES}_k(0^{64})
\]

\[
  f(k) = \mathrm{AES}_k(0^{128})
\]

\subsection{Limitations of OWFs}

\begin{enumerate}
  \item if \(f:\{0, 1\}^n \to \{0, 1\}^\ell\) is computable in time \(t_0\) then it is not \((O(2^n t_0), 1)\)-one-way
  % this means brute force
  \item \(f : \{0, 1\}^n \to \{0, 1\}^\ell\) cannot be \((O(n), \max \{\frac{1}{2^n}, \frac{1}{2^\ell}\})\)-one-way
\end{enumerate}

Consider a random guessing adversary: \(A(z)\) outputs \(y \in_R \{0, 1\}^n\):

\[
\begin{aligned}
  \Pr[A(f(x)) \in f^{-1}(f(x))] &= \Pr_{x,y}[y \in f^{-1}(f(x))] \\
  &= \Pr_{x,y}[f(x) = f(y)] \\
  & \ge \frac{1}{2^n}\text{ (prob that \(x = y\))}
\end{aligned}
\]

\[
\begin{aligned}
\Pr_{x,y}[f(x) = f(y)] &= \sum_{z \in \{0, 1\}^\ell} \Pr[f(x) = z \;\&\; f(y) = z] \\
&= \sum_{z \in \{0, 1\}^\ell} \Pr[f(x) = z] \cdot \Pr[f(y) = z] \\
&= \sum_{z \in \{0, 1\}^\ell} \Pr[f(x) = z]^2 \\
& \ge \frac{1}{2^\ell}
\end{aligned}
\]

This shows why it's a good idea for both the input and output to be long.

\section{Computational Number Theory}

\subsection{Sampling random prime numbers}

The naive approach is to simply randomly sample numbers and check if they are prime.
Fortunately, prime numbers are common, and we can efficiently check if they are prime.
As a matter of fact, the number of primes smaller than \(x\) is approximately \(\frac{x}{\ln x}\).

So how do we sample a random \(n\)-bit prime number in \(\polynomial(n)\) time?

\begin{enumerate}
  \item \(x \getsrandom \{2^{n-1}, \ldots, 2^n -1\}\)
  \begin{enumerate}
    \item \(x' \getsrandom \{0, 1\}^{n-1}\)
    \item \(x \gets 1x'\)
  \end{enumerate}
  \item Test if \(x\) is prime.
  \begin{enumerate}
    \item \(\Pr[\text{success}] = x / \ln x / x = \frac{1}{\ln 2^n} = \Omega(\frac{1}{n})\)
    \item \(L = [\text{\# attempts}] = O(n)\)
  \end{enumerate}
\end{enumerate}

\subsection{Modular arithmetic}

\(x \equiv y \mod n\) iff \(\divides{N}(x - y)\)

\[
  x \bmod N \overset{\text{def}}{=} [\text{unique \(x' \in \{0, \ldots, N-1\}\) \(x \equiv x' \pmod N\)}]
\]

\(\mathbb{Z}_N = \{0, \ldots, N-1\}\) with arithmetic (\(+, \cdot\)) \(\bmod N\)

\subsection{Greatest common divisor}

\(\gcd(a, b)\) every divisor of \(a\) and \(b\) divides \(\gcd(a,b)\).

For example, if we consider \(A = PQ\) and \(B=QR\), then \(\gcd(A, B) = Q\).

The extended GCD is: \(\forall x, y \in \mathbb{N}\; \exists a, b \in \mathbb{Z}\) such that \(ax+ by = gcd(x, y)\).
This can be found in polytime.

\[
  \mathbb{Z}_N^* = \{x \in \mathbb{Z}_N : \gcd(x, N) = 1 \}
\]

These are the elements in \(\mathbb{Z}_N\) with multiplicative inverses.
If \(\gcd(x, N) = 1\), then \(\exists a, b \in \mathbb{Z}\) such that \(ax + bN = 1\).
Take \(x^{-1} = a\) then \(ax = 1 \mod N\).

\paragraph{Example:}

\[
  N = 15 = 5 \cdot 3
\]
\[
  8 \in \mathbb{Z}_{15}^*
\]
\[
  8^{-1} = 2 \mod 15
\]
\[
  8 \cdot 2 = 16 = 1 \mod 15
\]

\[
\begin{aligned}
  a \cdot 8 + b \cdot 15 &= 1 \\
  2 \cdot 8 + -1 \cdot 15 &= 1
\end{aligned}
\]

\subsection{Euler phi function}

% Euler lived in Konigsberg, the seat of Prussian power
% he wrote over 1500 papers, and \(e^{\pi i} + 1 = 0\)

\[
  \phi(N) = |\mathbb{Z}_N^*| = N \cdot \prod_{\text{primes} \divides{P}N} (1 - \frac{1}{P}) \ge \frac{N}{6 \log \log N}
\]

For any prime number \(P\), \(\phi(P) = P - 1\).
For any prime numbers \(P, Q\):

\[
\begin{aligned}
  \phi(PQ) &= \phi(P) \cdot \phi(Q) = (P-1)(Q-1) \\
  &= N - 1 - (P - 1) - (Q - 1) \\
  &= PQ - P - Q + 1
\end{aligned}
\]

We don't have enough time to cover the rest of this section, so check the official lecture notes.

\subsection{Chinese Remainder Theorem}

Let \(N = PQ\) with \(\gcd(P, Q) = 1\).
Then the map \(x \mapsto (x \bmod P, x \bmod Q)\) from \(\mathbb{Z}_N\) to \(\mathbb{Z}_P \times \mathbb{Z}_Q\) is 1-1 and onto.

In particular, \(\forall y, z \in \mathbb{Z}_P \times \mathbb{Z}_q\), \(\nexists x \in \mathbb{Z}_N\) such that \(x \equiv y \mod P\) and \(x \equiv z \mod Q\).
Moreover, note that \(|\mathbb{Z}_P \times \mathbb{Z}_Q| = PQ = N = |\mathbb{Z}_N|\).

\paragraph{Proof}

We will describe the inverse mapping.
By extended gcd we can find \(a, b \in \mathbb{Z}\) such that \(\underbrace{aP}_{c}+\underbrace{bQ}_{d} = 1\).

\[
  \begin{matrix}
  c \equiv 1 \pmod P & d \equiv 0 \pmod P \\
  c \equiv 0 \pmod Q & d \equiv 1 \pmod Q
  \end{matrix}
\]

Inverse map \((y, z) \mapsto x = cy + dz \mod N\):

\[
\begin{aligned}
  cy + dz \equiv 1 \cdot y + 0 \cdot z \equiv y \pmod P \\
  cy + dz \equiv 0 \cdot y + 1 \cdot z \equiv z \pmod Q
\end{aligned}
\]

\section{Next lecture}

There will not be a lecture next week due to national elections.
There will not be a lecture for the two weeks after due to holidays.

\end{document}
