\documentclass[a4paper]{article}

\usepackage[T5]{fontenc}
\usepackage[utf8]{inputenc}
\usepackage{amsfonts}
\usepackage{mathtools}
\usepackage[iso]{datetime}
\usepackage{tabu}
\usepackage[colorlinks=true,urlcolor=blue,linkcolor=black]{hyperref}

\title{Statistics and Data Analysis\\\large Lecture 5}
\date{2016-12-06 \\ Last edited \currenttime\ \today}
\author{Lecture by Dr. Zohar Yakhini\\Typeset by Steven Karas}

\newenvironment{itemize*}%
  {\begin{itemize}%
    \setlength{\itemsep}{0pt}%
    \setlength{\parsep}{0pt}%
    \setlength{\parskip}{0pt}}%
  {\end{itemize}}

\newenvironment{enumerate*}%
  {\begin{enumerate}%
    \setlength{\itemsep}{0.5pt}%
    \setlength{\parsep}{0pt}%
    \setlength{\parskip}{0pt}}%
  {\end{enumerate}}

\newcommand{\mean}[1]{\mkern 1.5mu\overline{\mkern-1.5mu#1\mkern-1.5mu}\mkern 1.5mu}

\begin{document}

\maketitle

\paragraph{Agenda}
\begin{itemize*}
  \item Matlab intro
  \item HWA1 - Solution
  \item Class:Gauss, Inference, Central Limit Theorem
  \item HWA2 - Comments
\end{itemize*}

\section{Homework}

\subsection{Problem 1}
This is a binomial distribution. If we change the question to extend towards finding not one but more defective products, we want to find the $P(X\ge 3)$.

\subsection{Problem 3}
Example of solution where $E(X)=6$ and $E(Y)=7$:

\[X=\begin{cases}5 & \sim P(0.5) \\ 7 & \sim P(0.5)\end{cases}\]
\[Y=
\begin{cases}
  X=5 &
  \begin{cases}
    \frac{41}{5} & \sim P(0.5) \\
    \frac{43}{5} & \sim P(0.5)
  \end{cases}\\
  X=7 &
  \begin{cases}
    \frac{41}{7} & \sim P(0.5) \\
    \frac{43}{7} & \sim P(0.5)
  \end{cases}\\
\end{cases}
\]

Note that $P(x,y)\ne P(x)P(y)$.

\paragraph{Birthday paradox}
\[
P(\text{all unique}) =
P\left(
  \parbox{5em}{the $i$th is different from the $i-1$} \middle|
  \parbox{4em}{$i-1$ are different}
\right)=
\sum_{i=1}^{k} \left(1-\frac{i-1}{365}\right)
\]

\subsection{Problem 2}

\[Cost(k)=\text{Represents the cost of one night @$k$ tokens}\]

\[
E(Cost(k))=\frac{C}{365}\cdot k + \sum_{i=k+1}^{\infty} Poi(3,i)(i-k)\frac{C}{150}
\]
\[=\frac{C}{365}\cdot k-\frac{C}{150}\cdot k \underbrace{\sum_{i=k+1}^{\infty} Poi(3,1)}_{1-CDFPoi(3,k)} + \frac{C}{150}\sum_{i=k+1}^{\infty} i \cdot Poi(3,i)\]
\[E-\sum_{i=0}^{k}i\cdot Poi(3,i)\]

\subsubsection{Part 2}
\[\frac{c}{23} k - \frac{c}{20} k \sum_{i=4k+1}^{\infty} Poi(3000, i)\]
\[(1-CDFPoi(3000,i))\]
\[E-\sum_{i=0}^{4k}i\cdot Poi(3000, i)\]

\subsection{Problem}

\[\sum_{i=0}^{n} \binom{n}{k} i^2 p^i (1-p)^{n-i}=\underbrace{np(1-p)}_{V(X)}+\underbrace{(np)^2}_{E(X)^2}\]

He said that moving from what was written in the homework to what he wrote on the board is simple algebra, also noting that $E(X^2)-(E(X))^2$ is the definition of the left side.

\paragraph{Extra}
Change the $i^2$ to $i^4$.

\section{Matlab}

\begin{itemize*}
  \item All IDC computer labs have Matlab installed
  \item In MyIDC > IDCApps
  \item 65 USD for student license
\end{itemize*}


\paragraph{gmdistribution}
gmdistribution is a 

\[f_Y(t)=\sum_{i=1}^{k} w_i f_{X_i}(t) \text{ where } X_i \sim N(\mu_i,\delta_i^2)\]
\[F_Y(t)=\sum_{i=1}^k w_i F_{X_i}(t)\]

\section{Central Limit Theorem}
Let $X_1,...,X_n$ be random variables sampled independently from the same distribution with mean $\mu$ and nonzero variance $\sigma^2$. Let $\mean{X_n}$ be the average of $X_1,...,X_n$.

\[\lim_{n\to\infty} \left( \frac{\sqrt{n}}{\sigma} (\mean{X_n}-\mu) \le x \right)=\Phi(x)\]

Where $\Phi(x)$ is the standard normal density function.

% TODO: fill in board stuff from video

\paragraph{Intuition}
For almost any distribution, if we sample it enough, the average distance of our samples from the mean is normally distributed.

\subsection{Example}
Let $X\sim U[0,1]$. Assume that we drew $n=12$ samples.

\paragraph{}
What is $P\left( |\mean{X_n} - \frac{1}{2}| \le 0.1 \right)$? One way is to compute the distribution of $\mean{X_n}$. Much easier is to use the CLT as such:

For $U\sim U([0,1])$, we get $E(U)=\frac{1}{2}$ and $Var(U)=\frac{1}{12}$. Assuming that $n=12$ is large enough, we use the CLT:

% TODO: fill in from slide 3.

\subsection{Sum of Poissons}
Recall that the sum of Poissons is Poisson. Therefore, when $\lambda$ is sufficiently large, we can write $Poi(\lambda)$ as the sum of $\lambda$ independent copies of $Poi(1)$.

\paragraph{Example}
\[Poi(3000)=\sum_{i=0}^{3000}\bar X_i\]

\[\frac{n\mean{X_n}-n\mu}{\sqrt{n}\sigma} \sim N(0,1)\]

\[CDFPoi(3000,3017) \sim \Phi\left(\frac{3017-3000}{\sqrt{3000}}\right)\]

\subsection{Confidence Intervals}
Given a "fair" coin that is tossed 100 times, and turns up heads 61 times. What is the probability of this event? Can we believe the "fairness" of the coin?

\paragraph{Using CLT:}
\[P\left( \frac{\sqrt{100}}{0.5} (\mean{X_{100}}-0.5) \le x \right) \cong \Phi(x)\]
\[P\left((\mean{X_{100}}-0.5) \le \frac{x}{20}\right) \cong \Phi(x)\]
\[P((\mean{X_{100}}- 0.5) \le 0.11) \cong \Phi(2.2)\]

Thus, the probability of a fair coin landing heads more than 60 times is $~0.023$ and we can decide if we want to trust the coin, or if we reject the null model. It's important to remember that the null model has several components:

\begin{itemize*}
  \item The coin is fair
  \item The same coin was tossed 100 times
  \item The tosses are independent of each other
\end{itemize*}

\paragraph{Continuing}
What if the coin turned up heads 54 times? We get this is $>0.15$.

\paragraph{Defn}
Given a predetermined desired confidence level, we can find a range around $\mu$, such that if we observe a sample outside this range, we should reject the null model. It is sometimes used in the inverse, by virtually rejecting all other options.

\subsubsection{Example: Cholesterol Levels}
In a survey in the early 2010s, the US HRSA found that the mean total cholesterol (TCh) level for males aged 20 and over is 211 mg/dL and the stddev is 46 mg/dL.

We are sampling 25 individuals in some specific hospital for performing a healthcare experiment. We measure the sample mean for TCh in this population and want to test whether TCh might be a confounding variable. We will reject the null hypothesis if the sampel mean falls outside an interval around $\mu=211$ where it should fall in 90\% of all cases.
We use the CLT to compute the interval such that:
\[P((\mean{X_{25}}-211) \le 9.2x)\cong\Phi(x)=0.95\]
We check our tables to get that $x$ is around 1.65. Thus:
\[15.2 \le (\mean{X_{25}}-211)\le 1.65\cdot 9.2=15.2\]

This gives us a confidence interval of $(196,226)$\footnote{In scipy: `norm.interval(0.9, loc=211, scale=46/math.sqrt(25)'} for rejecting the null hypothesis.

\subsection{Gene Regulation}
% TODO: fill in from video and slides

\subsection{Internet CTR}
Checking success of a banner ad. First, we assume clicks are independent.

% TODO: fill in from slides/video

\subsubsection{A}
If we repeat experiments in which 
\subsubsection{B}

\section{Log Normal Distribution}
A random variable $Y$ is said to have a log-normal distribution if its log has a normal guassian distribution.

$Y=e^{\mu+\sigma Z}$ where $Z$ is the standard normal.

Log-Normals are useful for intrinsically positive quantities. Note that the mean, median, and mode are all different, and the distribution has a heavy tail.

\section{Homework 2}
Due December 17th.

\subsection{Problem 3}
Let $X_0 \sim U\{-5,...,5\}$.\footnote{"Same as $X_0$" means resample}


\end{document}
