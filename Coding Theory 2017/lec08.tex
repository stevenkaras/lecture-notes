\documentclass{idc_msc}

\usepackage{tikz}
\usetikzlibrary{graphs,positioning,quotes}

\title{Coding Theory\\\large Lecture 8}
\date{2017-12-26 \\ Last edited \currenttime\ \today}
\author{Lecture by Dr. Elette Boyle\\Typeset by Steven Karas}

\newcommand{\Fq}[1][q]{{\mathbb{F}_{#1}}}
\DeclareMathOperator*{\Vol}{Vol}
\DeclareMathOperator*{\RS}{RS}
\newcommand{\finiteuniform}[1][\mathbb{D}]{{\mathcal{U}_{#1}}}
\newcommand{\BSC}[1][p]{{\text{BSC}_{#1}}}

\AtEndDocument{\bibliographystyle{plain}\bibliography{biblio}{}}

\begin{document}

\maketitle

\paragraph{Disclaimer}

These lecture notes are based on the lecture for the course Coding Theory, taught by Dr. Elette Boyle at IDC Herzliyah in the fall semester of 2017/2018.
Sections may be based on the lecture notes written by Dr. Elette Boyle.

\paragraph{Agenda}

\begin{itemize}
  \item An efficient class of binary codes (expander codes)
  \begin{itemize}
    \item Today - construction + distance analysis
    \item Tomorrow - efficient decoding (linear time)
  \end{itemize}
\end{itemize}

\section{Review \& Background}

Last week, we covered an efficient decoding algorithm for RS codes.

RS codes in general have good distance and efficient decoding, but they rely on a large alphabet \(\Fq\).

Concatenated codes is a way to compose codes on top of each other, which is a typical strategy in implementing RS codes.

\subsection{Bipartite graphs}

A bipartite graph is a triple \(G=(L,R,E)\) of left vertices \(L\), right vertices \(R\), and edges \(E \subseteq L \times R\).

\subsection{Adjacency matrix}

The adjacency matrix of a graph is defined as \(A_G\) where the \((u,v)\) entry of \(A_G\) is 1 if \((u,v) \in E\).

\paragraph{Adjacency of bipartite graphs}

Because only a small subset of the generic adjacency matrix of a bipartite graph is populated, we can simplify it to be a \(|L| \times |R|\) matrix.

%TODO: put in an example of a simple graph and matrix

% \begin{center}
% \begin{tikzpicture}
%   [
%     bit/.style={draw,circle,minimum size=7mm,inner sep=0pt},
%     every edge/.style={draw,->,shorten >=0.5pt,>=stealth,semithick}
%   ]
%   \node [bit] (l1) {};
%   \node [bit] (l2) [below=of l1] {};
%   \node [bit] (l3) [below=of l2] {};
%   \node [bit] (l4) [below=of l3] {};
%   \node [bit] (r1) [right=of l1] {};
%   \node [bit] (r2) [below=of r1] {};
%   \node [bit] (r3) [below=of r2] {};

%   \graph[use existing nodes]{
%     l1 -> r2;
%     l2 -> r1;
%     l3 -> r1;
%     l3 -> r3;
%     l4 -> r3;
%   };
% \end{tikzpicture}
% \end{center}

% l1 - r2
% l2 - r1
% l3 - r1
% l3 - r3
% l4 - r3

% \[
% \begin{pmatrix}
% 0 & 1 & 0 \\
% 1 & 0 & 0 \\
% 1 & 0 & 1 \\
% 0 & 0 & 1
% \end{pmatrix}
% \]

\subsection{Factor Graph}

The adjacency matrix can be viewed as a parity-check matrix of a linear code.
Similarly, any binary linear code can view the parity check matrix as the adjacency matrix of a bipartite graph.
Note that the parity check matrix must be full rank, or the code is overly restrained with no codewords.

Specifically, given a parity check matrix \(H\), we can construct a bipartite graph \(G_H\) with adjacency matrix \(H\).

Or, given a bipartite graph \(G = (L, R, E)\) where \(|L| \ge |R|\), we can construct a linear code \(C(G)\) with parity check matrix \(A_G\).

\section{Expander Codes}

\subsection{Expander Graphs}

Informally, sparse graphs with good connectivity properties.

We will consider bipartite graphs with a linear number of edges (with relation to the number of vertices).

\paragraph{Left regularity}

A bipartite graph \(G = (L, R, E)\) is \(D\)-regular if every vertex in \(L\) has degree exactly \(D\).

\paragraph{Neighbor set}

For any vertex set \(S \subseteq L\), \(u \in R\) is a neighbor of \(S\) if \(\exists s \in S : (s,u) \in E\).
We denote by \(N(S)\) the set of all neighbors of \(S\), noting that \(N(S) \subseteq R\).

\paragraph{Unique Neighbor Set}

For any \(S \subseteq L\), \(u \in R\) is a unique neighbor of \(S\) if there is exactly one edge that connects \(u\) to \(S\).
Denote \(U(S)\) as the set of all unique neighbors of \(S\).

\paragraph{Bipartite Expander}

A bipartite graph where all subsets of \(L\) up to some size where \(|U(S)| > |S|\).

Formally, an \((n, m, D, \gamma, \alpha)\) bipartite expander is a \(D\)-regular bipartite graph \(G = (L, R, E)\) where \(|L|=n\), \(|R| = m\) such that for any \(S \subseteq L\), it holds that \(|S| \le \gamma n\) and \(|N(S)| \ge \alpha |S|\).

As such, \(\frac{1}{n} \le \gamma \le 1\).
Note that we trivially get that \((n, m, D, \frac{1}{n}, D)\) is an expander because each subset trivially expands to \(D\) vertices.
Similarly, \(0 \le \alpha \le D\).

Note that \(\gamma\) gives a measure of how small the expanding sets are, where a bigger \(\gamma\) is stronger.
Similarly, \(\alpha\) is the expansion factor which gives a measure of how much the sets expand.

\paragraph{Existence}

We can achieve expansion arbitrarily close to \(D\).

For any \(\varepsilon > 0\) and \(n \ge m\), there exists some bipartite expander \((n, m, D, \gamma, D(1-\varepsilon))\) where \(D = \Theta\left(\frac{\log(n/m)}{\varepsilon}\right)\) and \(\gamma = \Theta\left(\frac{\varepsilon m}{D n}\right)\).

The expansion factor \(\alpha = D(1 - \varepsilon)\) can get arbitrarily close to \(D\), at the expense of a larger degree \(D = \Theta\left(\frac{\log(n/m)}{\varepsilon}\right)\).

Note that we constrain this to \(n \ge m\). It is trivial to construct such an expander for the case where \(m = Dn\) by simply mapping to disjoint sets in \(R\).

A trivial lower bound on an achievable \(m\) is \(\gamma n D (1 - \varepsilon)\) as sets of size \(\gamma n\) must expand by a factor of \(\alpha\). Our proof will show for an \(m \sim \frac{1}{\varepsilon}\) times longer.

\paragraph{Lemma - Useful Property}

Let \(G=(L, R, E)\) be a \((n,m,D,\gamma, D(1-\varepsilon))\) bipartite expander with \(\varepsilon < 1/2\).

We want to show that for any \(S \subseteq L\) of size \(|S| \le \gamma n\), it holds that \(|U(S)| \ge D (1 - 2 \varepsilon)|S|\).

Recall that \(|N(S)| \ge D(1-\varepsilon)|S|\).
Moreover, \(D\)-regularity gives us that the total number of edges between \(S\) and \(N(S)\) is exactly \(D|S|\).

We use \(|N(S)|\) edges to touch the neighbors (or they wouldn't be neighbors). This gives us at most \(D|S| - D(1-\varepsilon)|S|=\varepsilon D |S|\) remaining edges.
In the worst case, each remaining edge disqualifies a different vertex, which gives us \(|U(S)| \ge N(S) - \varepsilon D |S| = D (1-2\varepsilon)|S|\).


\clearpage
\subsection{Expander Codes}

Let \(G = (L,R,E)\) be a bipartite expander graph where \(|L| \ge |R|\).
Then \(C(G)\) is an expander code.

Note that for \(|L|=n\) and \(|R| = n-k\), then for \((c_1,\ldots,c_n) \in \{0,1\}^n\) is a codeword in \(C(G)\) iff \(\forall r \in R\):

\[\sum_{l \in S : r \in N(\{l\})} c_l = 0\]

We've already shown that the rate of \(C(G) = \frac{n-k}{n} = \frac{|L|-|R|}{|L|}\).

We want to show the distance.

\paragraph{Weak distance bound}

Let \(G\) be a \((n,m,D,\gamma, \overbrace{D(1-\varepsilon)}^{\alpha})\) bipartite expander where \(\varepsilon < 1/2\).
We claim that \(C_G\) is a \([n,k,\gamma n+1]_2\) code.

We get all our claims for free except that the distance is \(\gamma n + 1\).
Suppose for contradiction that the distance of \(C(G)\) is \(\le \gamma n\).
Because \(C_G\) is a linear code, there exists some codeword \(\vec{c} \in C(G)\) with weight \(\le \gamma n\).
Let \(S = \{l \in L \mid c_l \ne 0\}\) (the vertices where the codeword is nonzero). As such, \(|S| = w(\vec{c}) \le \gamma n\).
Because \(G\) is an expander, by lemma, it holds that:

\[|U(S)| \ge \underset{> 0}{D}\;\;\underset{>0;\varepsilon < \frac{1}{2}}{(1 - 2\varepsilon)}\;\;\underset{> 0; \vec{c}\ne\vec{0}}{|S|}\]

So there exists some vertex \(r \in R\) with \(r \in U(S)\).
However, this implies that the parity of \(r\) is 1, because \(S\) is the set of all left vertices which are 1, and there is exactly one edge to \(r\).
However, this would imply that the parity check defined by \(r\) is not satisfied by \(\vec{c}\) (as \(c_l \ne 0\)).
Therefore, \(\vec{c} \notin C(G)\).
This contradicts our assumption that the distance of \(C(G)\) is \(\le \gamma n\), and therefore the distance is \(\ge \gamma n + 1\).

\paragraph{A better distance bound}

Instead of considering codewords of weight only up to the expansion threshold of \(\gamma n\), we will leverage expansion of a subset of the corresponding nonzero positions \(S\).

Let \(G\) be a \((n, n-k, D, \gamma, D(1-\varepsilon))\) bipartite expander with \(\varepsilon < 1/2\).
We claim that \(C(G)\) has distance of at least \(2 \gamma (1 - \varepsilon) n\).

Suppose for contradiction that the distance is strictly less than \(2 \gamma (1 - \varepsilon) n\).
Therefore, there is a nonzero codeword \(\vec{0} \ne \vec{c} \in C(G)\) with weight \(w(\vec{c}) < 2 \gamma (1 - \varepsilon)n\).

Let \(S = \{l \in L \mid c_l \ne 0\}\) be the nonzero vertices of \(\vec{c}\).
We will again argue that \(U(S) \ne \emptyset\) and then use the above argument to contradict.

If \(|S| \le \gamma n\) then the previous theorem still holds.
So assume that \(T \subset S\) such that \(|T| = \gamma n\).
Therefore, \(|U(T)| \ge D(1-2\varepsilon)|T|\).
But, some of the vertices in \(U(T)\) may also have a neighbor in \(S \setminus T\).
We want to upper bound the number of neighbors of \(S \setminus T\).
Since the number of edges from \(S \setminus T\) is \(D |S \setminus T|\), it follows that:

\[
\begin{aligned}
|N(S \setminus T)|
&\le D|S \setminus T| \\
&= D(|S| - |T|) \\
&\le D(2 \gamma(1-\varepsilon)n - \gamma n) \\
&= D(\gamma (1 - 2 \varepsilon)n) \\
\end{aligned}
\]

Note that this implies that \(|U(S)| \ge \underset{\ge D(1-2\varepsilon)\gamma n}{|U(T)|} - \underset{< D(1-2\varepsilon)\gamma n}{|N(S \setminus T)|}\) (as we throw away any \(r \in U(T)\) with an additional edge to \(S \setminus T\)).
This implies that \(|U(S)| > 0\), from which we can repeat the argument from before.

\clearpage
\subsection{Encoding \& Decoding}

Because this is a linear code, we can simply find the generator matrix which gives us encoding in \(O(n^2)\) time by matrix multiplication.
For expander codes, we can do this in linear time\cite{sipser1996expander}.

Decoding can be done in linear time because of the sparsity of the parity check matrix.

% TODO: put in a drawing of a bipartite graph

\paragraph{Decoding at a high level}

Given a graph \(G = (L, R, E)\) and a received word \(\vec{y} \in \{0, 1\}^n\).

First, evaluate the right hand side parity checks. Some will be satisfied and some unsatisfied.

Next, for every \(l \in L\), count how many of its \(D\) neighboring checks are unsatisfied.

Until there does not exist \(l \in L\) with \(> \left\lfloor\frac{D}{2}\right\rfloor\) neighboring checks unsatisfied, flip \(y_l\).

Because in each iteration, the corresponding right side neighbors flip from satisfied to unsatisfied and vice versa.

\section{Next Time}

Proof of decoding convergence.
Proof of decoding correctness.
Linear time encoding.

\end{document}
