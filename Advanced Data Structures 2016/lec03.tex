\documentclass[a4paper]{article}

\title{Advanced Data Structures}
\date{2016-11-17}
\author{Lecture by Shay Mozes\\Typeset by Steven Karas}

\usepackage[T5]{fontenc}
\usepackage[utf8]{inputenc}
\usepackage{amsfonts}
\usepackage{mathtools}

\begin{document}

\maketitle

\section{Fibonacci Heaps}

\paragraph{Design}
List of heap roots. Keep a reference to the smallest heap.

Heaps are implemented as a pointer to the largest child, and cyclical linked list between siblings.

\paragraph{Notation}

\[t(H)=\text{\# of trees in heap }H\]
\[m(H)=\text{\# of marked nodes in heap }H\]
\[\phi(H)=t(H)+2m(H)\]

\subsection{Delete-Min}
Promote the children of the removed root. Drop the marks from any immediate children\marginpar{In the slides, some roots are marked. They should not be.}. Merge heaps in order of root traversal until at most one heap of each degree is left.

\subsection{Decrease-Key}
Decrease the key. If the heap condition is violated, disconnect the subheap and merge it into the root list.

\paragraph{Unmarked Parent}
Mark the parent.

\paragraph{Marked Parent}
Disconnect the marked parent as a subheap and merge it into the root list. Mark the grandparent and repeat (this is sometimes referred to as cascade).

\paragraph{Complexity}
Let $c$ be the number of cascading grafts that were performed. $O(c)$.

\paragraph{Amortized Cost}
$O(1)$.

\[t(H')=t(H)+c\]
\[m(H')\le m(H)-c+2\]
\[\Delta\]
% TODO: fill in from slide 57

\subsection{Delete}
Decrease the key value to $-\infty$ and Delete-Min.

\subsection{Bounding Maximum Degree}
$D(N)=$ max degree in Fibonacci heap with N nodes.

\paragraph{Key Lemma}
$D(N) \le $

Key lemma. D(N)   log  N, where   = (1 +  5) / 2.
Corollary. Delete and Delete-min take O(log N) amortized time.
Lemma. Let x be a node with degree k, and let y1, . . . , yk denote the children of x in the order in which they were linked to x. Then:
degree(yi)  
 0 if i 1  i 2 if i 1
Proof.
When yi is linked to x, y1, . . . , yi-1 already linked to x,
  degree(x) =i-1
  degree(yi) = i - 1 since we only link nodes of equal degree
Since then, yi has lost at most one child
– otherwise it would have been cut from x Thus,degree(yi)=i-1 or i-2
\[D(N)=\]

% TODO: fill in from slide 59

\subsubsection{Lemma}
In a Fibonacci heap with N nodes, the maximum degree of any node is at most $\log_\phi N$, where $\phi=\frac{1+\sqrt{5}}{2}$

\paragraph{Proof}
Proof in video around 703pm, slide 60

Let $size(x)=\text{\# nodes in the subtree rooted at x}$.
For any node $x$, we want to show that $size(x)\ge \phi^{\text{degree}(x)}$.
Let $s_k$ be the minimum size of the tree rooted at 

\paragraph{Fibonacci's Sequence}

\[
F_k=
\begin{cases}
1 & \text{if }k=0 \\
2 & \text{if }k=1 \\
F_{k-1}+F_{k-2} & \text{if }k\ge2
\end{cases}
\]

\[F_k\ge\phi^k\text{, where }\phi=\frac{1+\sqrt{5}}{2}\]

For $k\ge2$, $F_k=2+\sum_{i=0}^{k-2}F_i$

% Which implies that TODO: fill in from slide 61

Formal proofs on slide 64.

\section{Hashing}
The purpose of Hashing is to manage a set $S$ of elements within some universe $U$.

\paragraph{Query}
Is the element $x\in U$ inside $S$?

\paragraph{Insertion/Deletion}
$S'=S\cup{x}$ and $S'=S\setminus x$

\paragraph{Formally}
Let $h(x):S\to [m]$, where $[m]={0,...,m-1}$.

\subsection{Chaining}
Colliding elements are stored in a list. Worst case behavior of $O(n)$.

\subsection{Perfect Hashing}

This is the point where I left the lecture for Thanksgiving dinner.

\end{document}
