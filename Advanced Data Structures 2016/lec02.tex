\documentclass[a4paper]{article}

\usepackage[T5]{fontenc}
\usepackage[utf8]{inputenc}
\usepackage{amsfonts}
\usepackage{mathtools}
\usepackage[iso]{datetime}
\usepackage{tabu}
\usepackage[colorlinks=true,urlcolor=blue,linkcolor=black]{hyperref}
\usepackage{listings}

\title{Advanced Data Structures\\\large Lecture 2}
\date{2016-11-17 \\ Last edited \currenttime\ \today}
\author{Lecture by Dr. Shay Mozes\\Typeset by Steven Karas}

\newenvironment{itemize*}%
  {\begin{itemize}%
    \setlength{\itemsep}{0pt}%
    \setlength{\parsep}{0pt}%
    \setlength{\parskip}{0pt}}%
  {\end{itemize}}

\newenvironment{enumerate*}%
  {\begin{enumerate}%
    \setlength{\itemsep}{0.5pt}%
    \setlength{\parsep}{0pt}%
    \setlength{\parskip}{0pt}}%
  {\end{enumerate}}

\begin{document}

\maketitle

\section{Overview of Heaps}

\begin{tabu} to \linewidth{|r|c|c|c|c|c|}
\hline
Operation & Linked List & Binary & Binomial & Fibonacci\footnote{amortized} & Relaxed \\
\hline
make-heap & 1 & 1 & 1 & 1 & 1 \\
\hline
insert & 1 & log N & log N & 1 & 1 \\
\hline
find-min & N & 1 & log N & 1 & 1 \\
\hline
delete-min & N & log N & log N & log N & log N \\
\hline
union & 1 & N & log N\footnotemark[1] & 1 & 1 \\
\hline
decrease-key & 1 & log N & log N & 1 & 1 \\
\hline
delete & N & log N & log N & log N & log N \\
\hline
is-empty & 1 & 1 & 1 & 1 & 1 \\
\hline
\end{tabu}

\subsection{Binary Heaps}
Binary heaps are a type of balanced binary tree that efficiently implements a total ordering. Heaps are always perfectly balanced trees. As such, the height of the tree is $\Theta(\log n)$

\subsubsection{Max-Heap Property}
For each node n in a binary heap, the value at n is at least as large as both its children. This is true for all nodes in the heap.

\subsubsection{Heap-Maximum}
The root of the heap is the largest value in the heap. $\Theta(1)$

\subsubsection{Max-Heap-Insert, Heap-Increase-Key}
$\Theta(\log n)$. Insert at next open leaf, and swap with parents until no longer larger.

\subsubsection{Heap-Extract-Max, Max-Heapify}
Removing the last leaf is $\Theta(1)$, and is trivially easy.

To remove the root, swap it with the last leaf, and remove the last leaf.

\paragraph{Max-Heapify}
To rebalance a heap, compare the root to each of its children, and swap with the largest child. Recurse until root is not swapped, or is a leaf node.

$\Theta(\log n)$

\subsubsection{Build-Max-Heap}
Naive heap creation takes at worst $\Theta(n log n)$, even though for random permutations it's $\Theta(n)$.

By building the heap from the bottom up, we can run Max-Heapify on each leaf node and recurse upwards through the heap.

Number of nodes with height $i$: $\frac{n}{2^{i+1}}$

\subsubsection{Decrease-Key}
Changing the priority of a specific element in the heap.

\subsubsection{Union}
Combine two heaps. If the heap is stored using pointers, this can be done in $O(\log n)*O(\log m)$.

\paragraph{Correctness}
Each node at height h can move down h levels:

\[\sum_{h=1}^{\log n} \frac{n}{2^h} \le n\sum_{h=1}^\infty 2^h=n\cdot(\frac{1}{2}+\frac{1}{4}+...)=n \]

\subsection{Application: Dijkstra}

Better explained elsewhere. CLRS does it wonderfully.

\paragraph{Complexity}

\[
n \cdot T(\text{Delete-Min}) + m \cdot T(\text{Decrease-Key})
\]

\begin{tabu} to \linewidth{|r|c|c|c|}
\hline
Heap & Delete-Min & Decrease-Key & Overall \\
\hline
Binary & TODO & TODO & $O((n+m)\log n)$ \\
\hline
Fibonacci & TODO & TODO & $O(n\log n + m)$ \\
\hline
\end{tabu}

\subsection{Application: Prim MST}
Given a weighted directed graph, we want a minimal set of edges for which the entire graph is connected.

\begin{lstlisting}[frame=L]
for all vertices in G:
  let v.key = +Inf, v.parent = NULL
Let s be a random vertex in G
s.key = 0
Q = priority queue over V wrt key
While Q is not empty:
  u = Q.DeleteMin
  for each neighbor v of u:
    if v.key > w(uv)
      Q.DecreaseKey(v, w(uv))
      v.parent = u
\end{lstlisting}

\paragraph{Complexity}

\[
n \cdot T(\text{Delete-Min}) + m \cdot T(\text{Decrease-Key})
\]

Note that this is the same as Dijkstra's Algorithm.

\subsection{Binomial Trees}
A binomial tree is defined recursively as a tree of binary trees.

\paragraph{}

\begin{itemize*}
\item Number of nodes = $2^k$
\item Height = $k$
\item Degree of root = $k$
\item Deleting root yields k binomial trees $B_0,...,B_{k-1}$
\end{itemize*}


\subsection{Binomial Heaps}
List of binomial trees that satisfy heap property. 0 or 1 

\subsubsection{Extract-Min-Key}
Minimum key is guaranteed to be in one of the roots. At most $\lfloor{\log_2n}\rfloor + 1$ binomial trees.

\subsubsection{Union}
Set smaller tree root as root of new combined tree, set other as other child. Takes $\log n$ because merging tree of order 0 into a fully packed heap means merging each tree.

\subsubsection{Delete-Min}
Find min in root list. Remove the root. Merge the children of the deleted root as if they were another heap.

\paragraph{Complexity}
$O(\log n)$ because $O(1)$ to remove the root, and $O(\log n)$ to merge.

\subsubsection{Decrease-Key}
Bubble the key up the tree if it's too small.

\paragraph{Complexity}
$O(\log n)$, because maximum node depth is $\log_2 n$

\subsubsection{Insert}
Merge in the element as if it were a binomial heap.

\paragraph{Complexity}
$O(\log n)$

\paragraph{Amortized Complexity}
$O(1)$ as per a binary counter from last week.


\subsection{Fibonacci Heaps}

\paragraph{Guidelines}
Be lazy. Force the user to make us do work. If we already have to work, then do the least amount possible to simplify the data structure.

\paragraph{Design}
Doubly linked ring of heaps\footnote{This isn't necessary, but whatever}. Keep a reference to the smallest heap.

\subsubsection{Union}
Combine the two lists of heaps, update the root. $O(1)$

\paragraph{Amortized Complexity}
Define a potential function $\phi(H)=t(H)=$ the number of roots in the heap.

\[amort(\text{Union})=O(1)+t(H_1)+t(H_2)-(t(H_1)+t(H_2))=O(1)\]

\subsubsection{Insert}
Treat the singular value as a heap, and merge. $O(1)$

\paragraph{Amortized Complexity}
\[amort(\text{Insert})=O(1)+t(H)+1-t(H)=O(1)\]

\subsubsection{Delete-Min}
Delete the minimal root. Put the children of the deleted root into the list.

While finding a new minimum, convert the fibonnacci heap into a binomial heap.

\paragraph{Complexity}
Worse, and yet...

\paragraph{Amortized Complexity}
\[amort(\text{Delete-Min})=O(1)+t(H)+\parbox{6em}{\# children of minimal heap}+O(\log n)-t(H)=O(\log n)\]

\subsubsection{Decrease-Key}
Described in terms of splicing subtrees to become roots. First subtree is ok, second is not.

If the heap property in v is not kept, detach it to become a root. However, if we do this, we may remove too many nodes from a tree (needs to be exponential to the order $k$). Our solution to this is to mark nodes that have already lost a child, and if we splice off the child of a marked node, we splice off the node itself.

Problem: What happens if we remove the bottom child in a tree full of marked nodes?
Solution: ensure that any node loses at most 2 children.

\paragraph{Amortized Complexity}
Due to this marking, we need to redefine the potential function, and repeat our analysis for all other actions:

$\phi'(H)=t(H)+2m(H)$ where $m(H)=\text{number of marked nodes}$ and $t(H)=\text{number of roots}$

\[amort(\text{Decrease-Key})=c+c+2(-c+1)=O(1)\]

where $c$ is the number of splices performed.

\subparagraph{Proof}
Let $v$ be a node in a binomial tree of degree $k$ with children $y_1,...,y_k$, ordered by insertion order. We want to show that for any $i \ge 2$, the degree of $y_i$ is at least 2.

Before $y_i$ was attached as a child of $v$, they have the same degree, meaning that $y_i$ has $i-1$ children at that moment. Since then, $y_i$ has lost at most 1 child.

\paragraph{}
Let $S_k$ be the minimal size of a tree of degree $k$. Note that $S_0=1$, and $S_1\ge2$.

\[S_k=2+S_0+S_1+...+S_{k-2}\]

Note that:

\[ S_k \ge F_k=F_{k-1}+F_{k-2} \ge \phi^k \]

where $\phi=\frac{1+\sqrt5}{2}=$ is the root of $x^2-x-1=0$, which because of its connection to Fibonacci is where this heap gets it name from.

The remainder of the proof is in the video, and Shay does a better job of it than I do.

\end{document}
