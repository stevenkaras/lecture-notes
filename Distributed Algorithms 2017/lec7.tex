\documentclass{idc_msc}

\title{Distributed Algorithms\\\large Lecture 7}
\date{2017-05-17 \\ Last edited \currenttime\ \today}
\author{Lecture by Dr. Gadi Taubenfeld\\Typeset by Steven Karas}

\begin{document}

\maketitle

\paragraph{Disclaimer}

These lecture notes are based on the lecture for the course Distributed Algorithms, taught by Dr. Gadi Taubenfeld at IDC Herzliyah in the spring semester of 2017.
Sections may be based on the lecture slides and accompanying book written by Dr. Gadi Taubenfeld.

\paragraph{Agenda}

\begin{itemize}
  \item Consensus
\end{itemize}

\section{Consensus}

Our model is:

Given a set of processes, where each process $p_i$ has an input value $in_i$.
Processes may crash (fail-stop).

\subsection{What is consensus?}

The processes reach consensus when all non-faulty processes decide on the same value $v$.

A valid consensus is one in which $v \in \{in_1,\ldots,in_n\}$.

There are variants for binary and multivalued consensus.

\subsection{Impossibility of Consensus}

There is no asynchronous algorithm that solves consensus using atomic read/write registers in the presence of a single faulty process.\footnote{A formal proof may be found in Chapter 9.1 of the book, and next week}

We can work around this with stronger primitives\footnote{We will use the shared memory model for the next few lectures until we discuss message passing}, timing assumptions, or randomization.

\subsection{3-state RMW register}

Given a 3-state RMW register $z$ with an initial value of $\bot$.
The first process $p_i$ to access $z$ sets it to $in_i$.
All other processes adopt this value as the decision value.

This algorithm is wait-free, which means it can tolerate the failure of any number of processes.

\subsection{2 processes, 3 bits}

Given 2 atomic RW registers $r_i$ and a single RMW bit $z$.
Each process sets $r_i := in_i$.
Each process tries to set $z$ from 0 to 1.
If the process $p_j$ succeeds, the decision value is set to $in_j$.

\subsection{2 processes, 2 RMW bits}

Given 2 RMW bits $x$ and $y$ with an initial value of 0.
If $x=1$, then $p_i$ decides 1 and halts. Otherwise, $p_i$ sets $x$ to $in_i$ and continues.
If $y=0$, then $p_i$ sets $y$ to 1, decides $in_i$, and halts. Otherwise, $p_i$ continues.
if $x=0$, then $p_i$ decides 0 and halts. Otherwise, $p_i$ decides $1 - in_i$.

This works because it uses $x$ to communicate the value of $in_i$ that should be decided, and $y$ is used to decide who decides\footnote{A full proof is just case analysis}.

\subsection{Impossible Consensus}

\paragraph{3 processes}

Impossible for any number of RMW and atomic registers.

There is no consensus algorithm for $n$ processes that can tolerate 2 faults for any $n$.

\paragraph{2 processes, any atomic registers}

Impossible to solve.

There is no consensus algorithm for $n$ processes that can tolerate 1 fault for any $n$.

\paragraph{$n$ processes, 4 RMW bits, at most 1 fault}

Non-wait free. Two processes run the above algorithm and the decider writes the result to a % TODO: finish this from the slides.

%ERRATA: slide 19 should read one, not two in the RMW remark

\paragraph{Crash-safe semaphores}

We cannot implement a semaphore $S$ for three or more processes using RMW bits and atomic.
Proof by contradiction: given such a semaphore, we can implement a consensus algorithm that is fault tolerant.

\subsection{Consensus without memory initialization}

Assume the following model:

\begin{itemize}
  \item A single RMW register
  \item the initial value of the register is not known
  \item Processes are asynchronous
  \item stop-failure
\end{itemize}

\paragraph{A solution for a fault-free model}
Increment the register $n$ times, and wait until the register reaches $n$ more than the initial value. Then set the register to $in_i + n + start + 1$. If the other processes see the register has grown by more, then they extract $in_i$ from the register and decide on it.

\paragraph{Bounded register size}
At least $2n + \max(in_i) + 1$.

\paragraph{Impossibility for $n/2$ failures}
If half the processes fail, then it is impossible to solve this problem.
Proof by contradiction: assume that there is such an algorithm.
Run this algorithm on half the processes with the input 0.
Run the algorithm again on the other half with their input set to 1.
Both groups have decided on different values, and therefore there cannot be such an algorithm.

\paragraph{An algorithm for $n/2-1$ failures}
Using a single RMW register with $3n$ values.

Use the MSB of the register as a separate bit, such that we can construct two rings with $1.5n$ values each.

Each process remembers its location on the current ring and increments the register on the ring.
It then waits for the location on the ring to be at least $n/2$ from its location. This process becomes the master.
The master flips the MSB (switching the active ring), and sets the value to $0$ if the consensus value is 0, and $n/2$ if it's 1.
If another process switched rings, then we decide on 0 or 1 based on the location on the ring.
If the location on the current ring has moved by more than $n$, then we decide based on the location on the ring.
The master keeps setting the location back to 0 or $n/2$.

\paragraph{Fault-free solution using 5 values}
Based on a See-saw barrier. Details on page 215 of the book.

\paragraph{Lower bound on number of values}
When there are faults, the lower bound on the number of values is $\Omega(n^{0.63})$

\subsection{Consensus using a shared queue}

Solution must be wait-free.
Can use many queues and atomic registers.

\paragraph{Observations}

\begin{itemize}
  \item With a peek operation: trivial for many operations
  \item Three processes - Impossible without peek
  \item 2 processes with an initialized queue - trivial without peek
  \item 2 processes with an empty queue - nontrivial without peek
\end{itemize}

\paragraph{2 processes with an initialized queue}

Start with 2 values in the queue: win and lose.
Each process dequeues a value. The winner sets the deciding value.

\paragraph{2 processes with an empty queue}

Algorithm designed by Roi Ramon in approx 2007.

Given a queue Q, initially empty, and an atomic bit $R$ with an initial value of $0$.

\begin{lstlisting}[frame=L,mathescape=true,title={For processes $i=1...n$}]
if $in_i = 0$:
  enqueue($Q$, $0$)
  if $R = 0$:
    decide($0$)
  else:
    if dequeue($Q$) = empty:
      decide(0)
    else:
      decide(1)
else:
  $R = 1$
  if dequeue($Q$) = empty:
    decide(1)
  else:
    decide(0)
\end{lstlisting}

\section{Next week}

\begin{itemize}
  \item Impossibility of Consensus - full proof
  \item Universality of Consensus
\end{itemize}

\end{document}
