\documentclass[a4paper]{article}

\usepackage[T5]{fontenc}
\usepackage[utf8]{inputenc}
\usepackage{amsfonts,amssymb}
\usepackage{mathtools}
\usepackage[iso]{datetime}
\usepackage{tabu}
\usepackage[colorlinks=true,urlcolor=blue,linkcolor=black]{hyperref}
\usepackage{tkz-graph}
\usepackage{listings}

\title{Advanced Algorithms\\\large Lecture 12}
\date{2017-01-29 \\ Last edited \currenttime\ \today}
\author{Lecture by Shay Mozes\\Typeset by Steven Karas}

\newenvironment{itemize*}%
  {\begin{itemize}%
    \setlength{\itemsep}{0pt}%
    \setlength{\parsep}{0pt}%
    \setlength{\parskip}{0pt}}%
  {\end{itemize}}

\newenvironment{enumerate*}%
  {\begin{enumerate}%
    \setlength{\itemsep}{0.5pt}%
    \setlength{\parsep}{0pt}%
    \setlength{\parskip}{0pt}}%
  {\end{enumerate}}

\begin{document}

\maketitle

\section{Sublinear Algorithms}
Streaming, sketching, property testing.

In many modern applications, the input is huge, and we can't possibly store or process all of the data.
Even if we can store it, we may not be able to process it in a reasonable time.
We still want to perform meaningful computations, so we need to change our computational model.

\subsection{Sublinear space - streaming model}
Input processed one element at a time.
Space consumed should be far less than the size of the input (e.g. polylog).
Sometimes, we allow multiple passes over the data.

\paragraph{Example: Routing}
Internet backbone routers process millions of packets per second.
Want to compute statistics on network activity.
Can’t store all the data going through.

\paragraph{Example: Web Search}
Determine statistics for ad placement.
Huge database, constantly changing.
Want to compute in single (or few) passes.

\paragraph{Similarities to Online algorithms}
We get data one element at a time, and the future is unknown.
Usually, we will analyze the worst-case stream.

\paragraph{Differences from Online algorithms}
Can delay decisions until output.
Memory is limited, and is sublinear by definition.

\paragraph{Example: Estimating Frequency}
Given an input $x_1,...,x_n$, provide $F(t)$ which is the number of times $t$ appears in the input.
We can think of $F(t)$ as a vector of length $n$ where $F(t) \in [m]$.
But this solution is too inefficient for large $m=poly(n)$.

We want to keep a "sketch" of data.

\subsubsection{Misra-Gries}
Invented in \href{http://www.mathcs.emory.edu/~cheung/papers/StreamDB/Frequency-count/1982-Misra-FindRepeatedElements.pdf}{1982 by Misra and Gries}.
Maintain a set $S$ of $k$ counters.
If $x_i \in S$: increment counter for $x_i$.
If there is space left in $S$, add $x_i$ to $S$.
Otherwise, decrement all counters in $S$ and remove any counters that reach 0.

$F(t)$ is a counter for $t$ if $t \in S$, otherwise 0.

\paragraph{Space}
\[k \cdot (\log n + \log m) = O(k \log n)\]

\paragraph{Accuracy}
Let $F'(t)$ be the frequency as reported by this algorithm. Let $F(t)$ be the actual frequency.

\[F'(t) \le F(t)\]
$F'(t) < F(t)$ can happen in two cases:

\begin{enumerate}
  \item $t$ wasn't in $S$ when it appeared and $S$ was full.
  \item $t$ was in $S$ when some $s \notin S$ caused $F'(t)$ to be decremented.
\end{enumerate}

Both cases cause $k$ decrements, and the total number of occurrences is at most $m/k$.
Therefore, because the sum of decrements cannot be more than the increments, it follows that the additive error is at most $m/k$.
\footnote{A visual example was made on the whiteboard}

\subsubsection{Turnstile Model}
Input consists of pairs: $x_i, c_i$, where $c_i$ indicates the change in $x_i$. $c_i$ can be negative.
Total at every point must be non-negative.

We will show a randomized algorithm with an additive error of $\varepsilon \cdot |F|_1$, where $|F|_1=\sum |c_i|$ guaranteed with probability $1-\delta$.

\subsection{Count Min Sketch}
Invented by \href{http://dimacs.rutgers.edu/~graham/pubs/papers/cm-full.pdf}{Cormode and Muthukrishnan '05}.

Maintain $t$ arrays with $k$ counters.
For array $i$ choose a random hash function $h_i:[n] \to [k]$.
On input $(x_j,c_j)$, for $i=1...t$:
add $c_j$ to entry $h_i(x_j)$ of array $i$.

\[F'(x)=\min_i A[i,h_i(x)]\]

\paragraph{Accuracy}
Let $k=2/ \varepsilon$ and $t=\log (1/\delta)$.

\[\Pr[|F'(x)-F(x)|>\varepsilon \cdot |F|_1] < \delta\]

% Proof done in class
\subparagraph{Proof}
Construct an element $a$ such that $X_i=\underbrace{A[i,h_i(a)]}_{\text{The counter for a in row i}}-F(a)$ is the error for $a$.
Let $Y_{ib}=\{h_i(a)=h_i(b)\}$ be a random variable that indicates a hash collision between $a$ and $b$ in row $i$.
We used random hash functions $h_j$, so it follows that $E[Y_{ib}]=\frac{1}{k}$.

\[E[X_i]=\sum_{b\ne a} \Pr[Y_{ib}=1]F(b)=\sum_{b\ne a} E[Y_{ib}]F(b) \le \frac{1}{k} \sum_b F(b) \le \frac{1}{k}|F|_1\]
\[\Pr[X_i > \overbrace{\frac{2}{k}}^{=\varepsilon} |F|_1] \le \frac{E[X_i]}{\frac{2}{k}|F|_1} \le \frac{\frac{1}{k}|F|_1}{\frac{2}{k}|F|_1}=\frac{1}{2} \]

\paragraph{Space}

\[O(t \cdot k \cdot \log |F|_1 + t \cdot \log k \cdot n)=\Omega(n)\]

\subsubsection{Pairwise-Independent Hashing}
Let $H$ be a family of functions from $[0...n-1]$ to a $[0...k-1]$.
We say that $H$ is pairwise independent if for every $x\ne y \in [0...n-1]$ and $c,d\in [0...k-1]$:
When $h$ is chosen uniformly from $H$, $\Pr_h[h(x)=c \text{ AND } h(y)=d]=\frac{1}{k^2}$.

There is a family of pairwise-independent functions that require just $\log(n)+\log(k)$ space.

Note that $\Pr_h[h(x) = h(y)] = 1/k$, and therefore Count-min sketches are efficient.

\paragraph{Construction}
Assume that $k$ is prime and $n=k^r$.
Think of $u\in [0...n-1]$ as an integer in base $k$ such that $u=\sum_i u_i k^i$.
The following family $H$ is pairwise independent:

\[h_{\bar{a},b}(u)=b+\sum_{i=0}^{r-1}a_i u_i \mod k \quad;\quad 0 \le a_i,b \le k-1\]

% a comment was made regarding the number of such functions, but I didn't catch it.

To represent one function, we need to store $r+1=O(\log n / \log k)$ numbers, each in $\log k$ bits.

\paragraph{Proof}
We need to compute the probability that $h(u)=c$ and $h(v)=d$.
If $u\ne v$, then there is some $i$ such that $u_j \ne v_i$.
Given any choice of the other $a_j$'s, we have:

\[a_i u_i+b=\left(c-\sum_{j\ne i} a_j u_j\right) \mod k\]
\[a_i v_i+b=\left(d-\sum_{j\ne i} a_j v_j\right) \mod k\]

This set of equations has a unique solution ($k$ is prime).
This means there are $k^{n-1}$ choices for the other $a_j$'s and $k^{r+1}$ choices overall.

\[\Pr[h(u)=c\text{ AND }h(v)=d]=\frac{1}{k^2}\]

\subsubsection{Estimating Distinct Elements}
We skipped this section, because we wanted to see a non-streaming algorithm.

\subsubsection{More streaming algos}

\begin{itemize*}
  \item Graph algorithms - input is sequence of edges; questions such as "how many triangles", "max-weight matching"
  \item Geometric algorithms - input is sequence of points; questions such as "k-center clustering"
  \item Multiple passes
  \item Random Order
  \item and many more
\end{itemize*}

\subsection{Sublinear time}
We model this with us having random access to the input.
Time consumed should be small compared to the input (polylog).
Sample of the input (sketch) is used to deduce approximate answer.
Using this for decision problems is called property testing.

\subsubsection{Example: Counting the fraction of 1s}
Input is a binary string $s$ of $n$ bits.
We need to output the fraction $\mu$of 1s in $s$.

\paragraph{Algorithm}
Return the fraction of $\mu'$ of 1s in a sample of $k$ locations chosen uniformly and independently at random.

\[\Pr[|\mu'-\mu| \ge \varepsilon] \le 2e^{-2k\varepsilon^2}\quad\text{[Chernoff]}\]

\subsubsection{Example: Property Testing of Monotonicity}
A decision problem, where the input an array $s[]$ of distinct members.
We need to output an approximate decision of whether or not the array is sorted.

%TODO: insert formal definition from slide 27.

A natural algorithm for this is to use the birthday paradox, but it fails horribly.

\paragraph{Binary search}
Discovered by \href{http://www.cs.columbia.edu/~rocco/Teaching/S14/Readings/EKKRV-spot-checkers.pdf}{Ergun Kannan Kumar Ravi Rubinfeld Vishwanathan}.
Repeat $\Theta(1 / \varepsilon)$ times:
Pick an element $x$ uniformly at random.
Perform a binary search for $x$.
If $x$ was not found, reject.
If all $x$'s found, then accept.

\paragraph{Claim}
The entries for which binary search would succeed form a sorted sequence.
\footnote{There was a proof on the whiteboard, but I missed it}

% Assume the negative that there are less than $\varepsilon \cdot n$ elements for whom the binary search will fail.
% There are at least $(1-\varepsilon)n$ elements for whom the binary search will succeed.
% There is a sorted subsequence $ > (1-\varepsilon)n$.
% The array is not $\varepsilon$-far from being sorted.
% Each experiment has succeed rate of $\varepsilon$.

\subsubsection{Additional Property Testing Examples}

\begin{itemize*}
  \item Checking if an array is monotonously sorted
  \item Checking if a string is in a regular language
  \item Checking if a function is linear
  \item Checking if a graph is 3-colorable
\end{itemize*}

It's interesting to note that some NP-hard problems can be tested in sublinear time, which can save us a lot of time.

\section{Review}

%NOTE: there was a hint that the topics in the slides in red may be more important for the exam

\begin{itemize*}
  \item Stable marriage and induction
  \item Divide and conquer
  \item Approximation algorithms
  \item Structured inputs - dynamic programming and tree decompositions
  \item Parametrized complexity - FPT, Kernalization, branching
  \item Linear Programming - Approximation via relaxation, LP duality
  \item Randomized algorithms - Markov and Chernoff bounds, sampling
  \item Online algorithms
  \item Sublinear algorithms
\end{itemize*}

\end{document}
