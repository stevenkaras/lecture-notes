\documentclass{idc_msc}

\title{Seminar on Logic Mathematics and Philosophy \\\large Lecture 01}
\date{2018-10-14 \\ Last edited \currenttime\ \today}
\author{Lecture by Dr. Udi Boker\\Typeset by Steven Karas}

\AtEndDocument{\bibliographystyle{plain}\bibliography{biblio}{}}

\begin{document}

\maketitle

\nocite{benacerraf1983philosophy}

\paragraph{Disclaimer}

These lecture notes are based on the seminars on Logic, Mathematics, and Philosophy; led by Dr. Udi Boker at IDC Herzliyah in the fall semester of 2018/2019.
Sections may be based on the lecture slides prepared by Dr. Udi Boker.

\section{Course Admin}

As a seminar, the first two lectures will be covered by Dr. Boker.
The remaining lectures will be led by pairs or triplets of students.
There are a list of suggested topics but other proposals are welcome.

Attendance is mandatory.
If you can't make it to a lecture, points will be taken off as per \(2^{n+1}\) where n is the number of unexcused absences.
80\% of the grade is based on the lecture.
20\% of the grade is on homework.

\subsection{Lectures}

Each lecture is prepared by students, with two meetings with Dr. Boker two and one week before the lecture date.
As part of building the lecture, a homework assignment must also be prepared.

\section{History of Mathematics}

\subsection{Classical Greece}

Pythagoras (circa 550 BCE) was a member of a cult that believed natural events were built out of ratios of natural numbers.
Contemporaneously, there were natural numbers and simple arithmetic.
Pythagoras proved that \(a^2 + b^2 = c^2\) for right triangles.
This was a problem, because it implied the existence of irrational numbers, and that arithmetic is not a good model of mathematics.

Zeno (circa 450 BCE) proposed many paradoxes.
He posed the following as his "classic" one: if a turtle runs half as fast as a rabbit, but starts a meter ahead, at what distance does the rabbit pass the turtle?
This was presented as an infinite sum of fractions.

Plato\footnote{Whitehead said that all of modern western philosophy is just margin notes for classical philosophy} (circa 400 BCE) was the head of the eponymous Platonic school of philosophy.
Part of his philosophy was that we live in many worlds some of which are ideal worlds, including among them mathematics.

Aristotle was a student of Plato and advisor to Alexander the Great.
His primary contribution was to create a formal system of logic.
His approach was that the ideal worlds are based on the material world we experience.

Euclid (circa 300 BCE) was a mathematician who wrote "Elements", which established the basis of proven mathematics and geometry.
He proved that there are a potentially infinite\footnote{potential infinity just means that the number of something is unbound} number of prime numbers.

Diophantus (circa 200 CE) laid down the basis of algebra, which extends basic number theory to variables and the generic form of arithmetic.

There are many more major players in this period, but we have limited time and will focus on just a few key players.

\subsection{Middle Ages}

Brahmagupta (7th century) in India introduced the concept of zero and decimal numeric representation\footnote{both of these were invented by others independently, but we'll focus on him for the purpose of the course}.

Muhammad ibn Musa al-Khwarizmi (9th century) in Baghdad documented algebra which is named after him.

Fibonacci (12th century) heavy pushed the concept of zero, positional notation, and algebra to the Europeans.

In this period, real numbers were used, but not formally or in an organized manner.

\subsection{17th Century}

The rationalist school of philosophy attempted to look at everything in a mathematical way.

The empirical school looked at the world as something that is sensed, and experienced.
They rejected the concept of mathematical proof and preferred experimental confirmation.
Knowledge is a posteriori, not a priori.

Galileo was an empiricist.

Renee Descartes is often treated as a rationalist, but tried to reconcile the two schools.
He combined geometry and algebra, and got analytical geometry.
He gave the name of real numbers, but did not give a formal definition.

Isaac Newton invented calculus.

Gottfried Leibniz also invented calculus independently.

In this period, they used infinitesimal numbers to solve formulas, which were then thrown out to get an exact solution.

\subsection{18th century}

George Berkeley had a big problem with infinitesimal numbers, and did not like the application of potential infinity.

Immanuel Kant mixed the two schools of rationalism and empiricism.
He claimed that there were analytical (a priori) statements - those statements which do not require experimental proof.
He also presented synthetic (a posteriori) statements - those statements which add knowledge.
He presented mathematical theorems as synthetic a priori statements in that we do not need to empirically prove them, but they are not analytical.

A priori truths are derived from the truth of their structure.
Synthetic statements are those that give knowledge.
A priori synthetic statements are those that do not require experimentation to prove their truth. E.g. \(5 + 7 = 12\).
A posteriori synthetic statements are those that do require experimentation.

Baruch Spinoza was a rationalist who in his "Ethics" attempted to apply mathematical logic to derive philosophical truth.

Pierre de Fermat was another important figure in this period.

\subsection{Early 19th Century}

Augustin-Louis Cauchy dealt with infinitesimal numbers and defined calculus in terms of limits.

Karl Weierstrass also defined calculus in terms of limits.

The majority of mathematical work up until this period was based in euclidean geometry.

\subsection{Mid 19th Century}

George Boole invented the notation for symbolic boolean logic.

Augustus De Morgan also worked on symbolic logic.

Carl Friedrich Gauss was also a major player in this period.

Nikolai Lobachevsky introduced non-euclidean geometries.

Bernhard Riemann also dealt with non-euclidean geometries.
He rejected the parallel line axiom and showed that different geometries are possible and useful, such as the surface of a sphere.

This was a major development because they were able to derive much of modern geometry and interesting proofs without using the axiom of parallel lines.
This was a big deal because there were not many attempts to derive proofs without using all the axioms.
It effectively caused doubt in the a posteriori truth of euclidean geometry.

\subsection{Late 19th Century}

Georg Cantor developed naive set theory with the help of Richard Dedekind.
He was able to prove the existence of different types of infinity, and that the real numbers are "bigger" than the natural numbers.
His famous theorem was that the cardinality of subsets of a set is greater than the cardinality of a set.

Leopold Kronecker had harsh criticism of Cantor regarding the use of actual infinity, as it was "divine".

Richard Dedekind formally defined the real numbers using an infinite set and collaborated with Cantor on the initial development of set theory.

David Hilbert proved the consistency of euclidean geometry using the real number model.
This was a big deal because he formally proved the 5 axioms are consistent.
He used a formalist approach that focused on consistent axioms.
He believed in finitism, which rejects the existence of nonfinite objects such as concrete infinite sets.
He spent a lot of time attempting to use Peano Arithmetic to imply the correctness/consistency of higher-order logics.

Friedrich Ludwig Gottlob Frege built first order logic and founded the philosophical school of logicism.
This school was at odds with Kant, as they claimed that mathematics is analytical.

Giuseppe Peano invented a good notation for first order logic.
He also extended it to what is called "Peano Arithmetic" which was a formal basis for arithmetic and algebra.

% Henri Poincar\'e

\subsection{20th Century}

Bertrand Russell sent his paradox to Frege, in that a set of all sets that do not belong to themselves implies that it includes itself iff it does not:

Let \(A\) be a set.
Define \(A \notin A\) as a property.
Let \(S=\{A \mid A \notin A\}\).
\(S \in S\)?

If \(S \in S\), then \(S \notin S\), by definition.
If \(S \notin S\), then \(S \in S\), by definition.

This made a lot of people very angry, because it showed that naive set theory has contradictions.
Russell himself went towards Type Theory to find a solution.

Alfred Whitehead collaborated with Russell on "Principia Mathematica", which derived all mathematics from type theory.
The notation was overly complex and while not directly used did inspire many other approaches.

Ernst Zermelo and Avraham Fraenkel built axiomatic (ZFC) set theory, which explicitly disallowed Russell's paradox and is generally regarded as a good basis for modern mathematics.

Luitzen Egbertus Jan Brouwer was the founder of mathematical intuitionism which disallows contradictions and allows abstaining, constructing mathematics from positive proofs only.

Kurt G\"odel proved that any mathematical system is either inconsistent or incomplete.
He also showed that the continuum hypothesis cannot be implied from ZFC.

Paul Cohen was a formalist who showed\cite{cohen1963independence} that the continuum hypothesis is independent from ZFC.

Predicativism is the school of philosophy that accepts the natural numbers but disallows self-reference.

Alan Turing founded the field of Computer Science.
Alonzo Church and Turing proved that all computational models are equivalent to Turing machines.

\end{document}
