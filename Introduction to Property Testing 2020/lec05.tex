\documentclass{idc_msc}

\title{Introduction to Property Testing \\\large Lecture 5}
\date{2020-05-20 \\ Last edited \currenttime\ \today}
\author{Lecture by Dr. Reut Levi\\Typeset by Steven Karas}

\AtBeginDocument{\maketitle}
\AtEndDocument{\vfill\bibliographystyle{plain}\bibliography{../biblio}{}}

\let\defeq\relax
\newcommand\defeq{\stackrel{\mathclap{\normalfont\mbox{def}}}{=}}

\begin{document}

\nocite{goldreich2017introduction}

\paragraph{Disclaimer}

These notes are based on the lectures for the course Introduction to Property Testing, taught by Dr. Reut Levi at IDC Herzliyah in the spring semester of 2019/2020.
Sections may be based on the lecture slides prepared by Dr. Reut Levi.

\section{Agenda}

  \begin{itemize}
    \item Monotonicity
    \item Graph properties
  \end{itemize}


\section{Testing Monotonicity}

Consider \(f : [n] \to R_n\), where \(R_n\) is an arbitrary totally ordered set.
w.l.o.g. we may assume that the values of \(f\) are distinct (break ties according to index - i.e. augment \(f(i)\) to \((f(i), i)\)).
Note that we can consider \(f\) as an array filled with values.

\paragraph{Algorithm 2}

\begin{enumerate}
  \item select \(i \in [n]\) u.a.r.
  \item find \(f(i)\) in the array by performing a binary search.
  % a graphical demonstration of a binary search was given which I felt was unneccessary to include here.
  \item accept iff \(f(i)\) is found
\end{enumerate}

This has query complexity \(1 + \lceil\log n\rceil\).

\paragraph{Correctness}

If our algorithm accepts w.p. \(\ge 1 - \delta\), then \(f\) is \(\delta\)-close to monotone.

Note that the only random choice is \(i\).
We define that \(i\) is good (w.r.t. \(f\)) if the execution on \(i\) accepts.
if \(i < j\) are both good, then \(f(i) < f(j)\).
Let \(t\) be the first location where the binary search for \(i\) and \(j\) take different halves.
If \(f(i) \le f(t)\) and \(f(j) > f(t)\), then it follows that \(f(i) < f(j)\).

Observe that the restriction of \(f\) to the set of good points is a monotone function.
If we correct \(f\) on the non-good points we obtain a monotone function.
There are \(\ge (1 - \delta) \cdot n\) good points thus \(f\) is \(\delta\)-close to monotone.

\paragraph{A related non-adaptive tester}

Observe that after selecting \(i \in [n]\), the choice of queries can be determined a priori.
If we search for the index \(i\) rather than the value, we can make these queries and check that there are no violations.
Correctness follows similarly to above.

\paragraph{2-query algorithm}

There exists a 2-query POT for monotonicity with detection probability \(\delta / \lceil \log_2 n \rceil\).

\begin{enumerate}
  \item select \(i \in [n]\) u.a.r.
  \item select one of the queries of the nonadaptive algorithm u.a.r.
  \item reject iff we see a violation.
\end{enumerate}

\clearpage
\section{Dense Graph Model}

% dense graphs have O(n^2) edges

In the adjacency matrix model\footnotemark (a.k.a. the dense graph model), a \(k\)-vertex graph \(G = ([k], E)\) is repesented by \(g: [k] \times [k] \to \{0, 1\}\) such that \(g(u, v) = 1\) iff \(\{u,
v\} \in E\).
\footnotetext{Sparse matrices are typically stored as lists of edges per vertex}
We define the distance between \(G\) and \(G'\) as represented by \(g\) and \(g'\), respectively:

\[
  \delta(G, G') = \frac{|\{(u, v) : g(u, v) \ne g'(u, v)\}|}{|V|^2}
\]

We say that graph properties are sets of graphs that are closed under isomorphism.
That is, \(\Pi\) is a graph property if \(\forall G = ([k], E)\) and every permutation of \([k]\) : \(G \in \Pi\) iff \(a(G) \in \Pi\) where \(a(G) \defeq ([k], \{ \{a(u), a(v)\} : \{u, v\} \in E \})\).
In simple terms, the labels of the vertices do not matter, only the structure of the edges.

\subsection{Testing bipartiteness}

We define a bipartite graph as a graph whose vertices can be partitioned into \(A, B\) such that every edge is in \(A \times B\).
% there were several diagrams showing bipartite graphs, but I don't have the time/patience to reproduce them here
\footnote{There is no POT with constant query complexity, and we will likely have this as a question as part of the next homework set (exercise 8.5 in the book).}

\paragraph{Algorithm}

Given an input \(k, \varepsilon\):

\begin{enumerate}
  \item u.a.r. select \(\tilde{O}(1 / \varepsilon^2)\) vertices denoted by \(R\).
  \item accept iff subgraph induced on \(R\) is bipartite.
\end{enumerate}

% there were more diagrams demonstrating the query principle

This algorithm rejects graphs that are \(\varepsilon\)-far from being bipartite with probability \(\ge 2/3\).

\paragraph{Correctness}

Assume \(G\) is \(\varepsilon\)-far from bipartite and we'll see that with probability \(\ge 2/3\), \(G([R])\) is not bipartite.
View \(R\) as a union of \(U\) and \(S\) (disjoint) where \(t = |U| = O(\frac{\log\frac{1}{\varepsilon}}{\varepsilon})\) and \(m = |S| = O(\frac{t}{\varepsilon})\).
Consider all 2-partitions of \(U\).
For each 2-partition \(U_1, U_2\) of \(U\) consider the partial 2-partition of \([k]\):
One side: all neighbors of \(U_1\) are opposite to \(U_1\).
The other side: all neighbors of \(U_2\) are opposite to \(U_2\).
If \(v\) is adjacent to both \(U_1\) and \(U_2\) place it opposite to \(U_2\).

Denote \(N(v) \defeq \{u : \{u, v\} \in E\}\) and \(N(X) \defeq \bigcup_{v \in X} N(v)\).

Given a 2-partition \(U_1, U_2\) of \(U\), define a possibly partial 2-partition of \([k]\): \(V_1, V_2\) where \(V_1 = N(U_2)\) and \(V_2 = N(U_1) \setminus V_1\).
% a diagram shows the potential overlap of V_1, V_2

Define a vertex \(v\) as being high-degree if its degree \(\ge \frac{\varepsilon k}{6}\).
\(U\) is good if all but at most \(\frac{\varepsilon k}{6}\) of high-degree vertices are in \(N(U)\).
With probability \(\ge 5/6\), \(U\) is good.
Proof follows from any \(v\) that has high degree \(\Pr[N(v) \cap U = \emptyset] = (1-(\varepsilon / 6))^t < \varepsilon / 36\) (due to Markov's inequality).


\end{document}
